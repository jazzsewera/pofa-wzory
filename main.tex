\documentclass[12pt]{article}
\usepackage[margin=0.4in]{geometry} 
\usepackage{polski}
\usepackage[polish]{babel}
\usepackage[utf8]{inputenc}
\pagenumbering{gobble}
\usepackage{amsmath}
\usepackage{amsthm}
\let\lll\undefined
\usepackage{amssymb}
\usepackage{amsfonts}
\usepackage{esint}
\usepackage{multicol}
\usepackage{xcolor}
\usepackage{lmodern}
\usepackage[T1]{fontenc}
\setlength{\columnseprule}{1pt}
\def\columnseprulecolor{\color{lightgray}}
\DeclareMathOperator{\der}{\operatorname{d}\!}
\newenvironment{bottompar}{\par\vspace*{\fill}}{\clearpage}
\newcommand{\grayrule}{{\color{lightgray} \hrule}}
\definecolor{m-gray}{gray}{0.4}
%%%%%%%%%%%%%%%%%%%%%%%%%%%%%%%%%%%%%%%%%%%%%%%%%%%%%%%%%%%%%%%%%%%%%%%%%%%%%%%%%

%%%%%%%%%%%%%%%%%%%%%%%%%%%%%%%%%%%%%%
%Do not alter this block.
\begin{document}
%%%%%%%%%%%%%%%%%%%%%%%%%%%%%%%%%%%%%%

\begin{multicols}{3}

% jednostki
\begin{multicols}{2}

\begin{equation*}
    \begin{split}
        \left[\Vec{E}\right] &= \frac{V}{m} \\
        \left[\Vec{H}\right] &= \frac{A}{m} \\
        \left[\Vec{D}\right] &= \frac{C}{m^2} \\
        \left[\Vec{B}\right] &= \frac{Vs}{m^2} \\
              \left[w\right] &= \frac{J}{m^3}
    \end{split}
\end{equation*}

\begin{equation*}
    \begin{split}
        \left[\Vec{J}\right] &= \frac{A}{m^2} \\
        \left[\overline{\overline{\varepsilon}}\right] &= \frac{F}{m} \\
        \left[\overline{\overline{\mu}}\right] &= \frac{H}{m} \\
        \left[\overline{\overline{\sigma}}\right] &= \frac{S}{m} \\
        \left[S\right] &= \frac{W}{m^2}
    \end{split}
\end{equation*}

\end{multicols}

% stałe w próżni
\begin{equation*}
    \begin{split}
        \varepsilon_0 &= \frac{10^{-9}}{36 \pi} \, \frac{F}{m} \\
        \mu_0 &= 4 \pi \cdot 10^{-7} \, \frac{H}{m} \\
        \rho_0 &= 0 \\
        j_0 &= 0
    \end{split}
\end{equation*}

\grayrule

\begin{equation*}
    \begin{split}
        c &= \frac{1}{\sqrt{\mu_0 \varepsilon_0}} \approx 3 \cdot 10^8 \frac{m}{s} \\
        v &= \frac{1}{\sqrt{\mu \varepsilon}} = \frac{c}{\sqrt{\mu_r \varepsilon_r}}
    \end{split}
\end{equation*}

\grayrule

% długość fali i częstotliwość
\begin{equation*}
    \lambda = \frac{c}{f}
\end{equation*}

\begin{equation*}
    f = \frac{1}{T}
\end{equation*}

\grayrule

% indukcje i pola
\begin{equation*}
    \Vec{D} = \overline{\overline{\varepsilon}} \cdot \Vec{E}
\end{equation*}

\begin{equation*}
    \Vec{B} = \overline{\overline{\mu}} \cdot \Vec{H}
\end{equation*}

\begin{equation*}
    \Vec{J} = \overline{\overline{\sigma}} \cdot \Vec{E}
\end{equation*}

\grayrule

% mnożenie wektorów

\begin{equation*}
    \Vec{a} \cdot \Vec{b} = a_x b_x + a_y b_y + a_z b_z
\end{equation*}

\begin{equation*}
    \Vec{a} \times \Vec{b}
        = \begin{vmatrix}
        \Vec{i_x} & \Vec{i_y} & \Vec{i_z} \\
        a_x & a_y & a_z \\
        b_x & b_y & b_z
        \end{vmatrix}
\end{equation*}

% operatory na skalarach i polach wektorowych
\begin{equation*}
    \operatorname{grad} f = \nabla f = \left(
            \frac{\partial f}{\partial x},\,
            \frac{\partial f}{\partial y},\,
            \frac{\partial f}{\partial z}
        \right)
\end{equation*}

\begin{equation*}
    \operatorname{div} \Vec{F}
        = \nabla \cdot \Vec{F}
        = \frac{\partial F_x}{\partial x}
            + \frac{\partial F_y}{\partial y}
            + \frac{\partial F_z}{\partial z}
\end{equation*}

\begin{equation*}
    \mathrm{rot}(\Vec{F}) = \nabla \times \Vec{F}
        = \begin{vmatrix}
        \Vec{i_x} & \Vec{i_y} & \Vec{i_z} \\
        \frac{\partial}{\partial x} & \frac{\partial}{\partial y} & \frac{\partial}{\partial z} \\
        F_x & F_y & F_z
        \end{vmatrix}
\end{equation*}

\grayrule
%%%%%%%%%%%%%%%%%%%%%%%%%%%%%%%%%%%%%%%%%%%%%%%%%%%%%%%%%%%%%%%%%%%%%%%%%%%%%
{ \color{m-gray}

    \begin{equation*}
        \oint\limits_C \Vec{F}\,\der \Vec{l}
            = \iint\limits_{S(C)} \nabla \times \Vec{F}\,\der\Vec{a} 
    \end{equation*}
    
    \begin{equation*}
        \oiint\limits_S \Vec{F} \, \der \Vec{a} = \iiint\limits_{V(S)} \nabla \Vec{F} \, \der\tau
    \end{equation*}
    
    \grayrule
    
    \begin{equation*}
        \begin{split}
            \oint\limits_C \Vec{E} \, \der\Vec{l} &= - \frac{\der\varphi}{\der t}\\
            \oint\limits_C \Vec{E} \, \der\Vec{l}
                &= - \frac{\der}{\der t} \int\limits_S \Vec{B} \, \der\Vec{s}
        \end{split}
    \end{equation*}
    
    \begin{equation*}
        \begin{split}
            \oint\limits_C \Vec{B} \, \der\Vec{l}
                &= \mu I + \mu \varepsilon \frac{\der\Phi_E}{\der t}\\
            \oint\limits_C \Vec{B} \, \der\Vec{l}
                &= \mu I + \mu \varepsilon \frac{\der}{\der t} \int\limits_S \Vec{E} \, \der\Vec{s}
        \end{split}
    \end{equation*}
    
    \begin{equation*}
        \varepsilon \, \oint\limits_S \Vec{E} \, \der\Vec{s} = q
    \end{equation*}
    
    \begin{equation*}
        \oint\limits_S \Vec{B} \, \der\Vec{s} = 0
    \end{equation*}

}
%%%%%%%%%%%%%%%%%%%%%%%%%%%%%%%%%%%%%%%%%%%%%%%%%%%%%%%%%%%%%%%%%%%%%%%%%%%%%
\grayrule

\begin{equation*}
    \nabla \times \Vec{E} = - \frac{\partial \Vec{B}}{\partial t}
\end{equation*}

\begin{equation*}
    \nabla \times \Vec{B} = \mu \Vec{j} + \mu \varepsilon \frac{\partial \Vec{E}}{\partial t}
\end{equation*}

\begin{equation*}
    \nabla \times \Vec{H} = \Vec{J} + \frac{\partial \Vec{D}}{\partial t}
\end{equation*}

\begin{equation*}
    \varepsilon \nabla \cdot \Vec{E} = \rho
\end{equation*}

\begin{equation*}
    \nabla \cdot \Vec{D} = \rho
\end{equation*}

\begin{equation*}
    \nabla \cdot \Vec{B} = 0
\end{equation*}

\grayrule

\begin{equation*}
    \Vec{J_D} = \frac{\partial \Vec{D}}{\partial t}
        = \varepsilon \frac{\partial \Vec{E}}{\partial t}
\end{equation*}

\grayrule

\begin{equation*}
    \frac{\partial \Vec{E}}{\partial t}
        = \frac{\partial (E \cdot e^{j \omega t})}{\partial t}
        = j \omega \cdot E \cdot e^{j \omega t}
\end{equation*}

\grayrule

\begin{equation*}
    \begin{split}
        \Vec{n} (\Vec{D_2} - \Vec{D_1}) &= \rho_s \\
        \Vec{n} (\Vec{B_2} - \Vec{B_1}) &= 0 \\
        \Vec{n} \times (\Vec{E_2} - \Vec{E_1}) &= 0 \\
        \Vec{n} \times (\Vec{H_2} - \Vec{H_1}) &= \Vec{J_s}
    \end{split}
\end{equation*}

\grayrule {\color{darkgray}\hrule} \grayrule

\begin{equation*}
    \omega = 2 \pi f
\end{equation*}
\begin{equation*}
    \lambda = \frac{2 \pi}{\beta}
\end{equation*}
\begin{equation*}
    \tan \delta = \frac{\sigma}{\omega \varepsilon}
\end{equation*}
\begin{equation*}
    Z = \sqrt{\frac{j \omega \mu}{\sigma + j \omega \varepsilon}}
\end{equation*}
\begin{equation*}
    \tan \delta = 0 \implies Z = \sqrt{\frac{\mu_r}{\varepsilon_r}} \cdot Z_0
\end{equation*}
\begin{equation*}
    \begin{split}
        \gamma &= \alpha + j \beta \\
               &= \sqrt{j \omega \mu (\sigma + j \omega \varepsilon)} \\
               &= j \omega \sqrt{\mu \varepsilon (1 - j \tan \delta)}
    \end{split}
\end{equation*}

\grayrule

\begin{equation*}
    \begin{split}
        \tan \delta &\ll 1 \implies \\
        \alpha &\approx \frac{\sigma}{2} \sqrt{\frac{\mu}{\varepsilon}} \\
        \beta &\approx \omega \sqrt{\mu \varepsilon} \\
        Z &\approx \sqrt{\frac{\mu}{\varepsilon}} \cdot e^{j \frac{\delta}{2}}
    \end{split}
\end{equation*}
\grayrule
\begin{equation*}
    \begin{split}
        \tan \delta &\gg 1 \implies \\
        \alpha &\approx \beta \approx \sqrt{\frac{\omega \mu \sigma}{2}} \\
        Z &\approx \sqrt{\frac{\omega \mu}{\sigma}} \cdot e^{j \frac{\pi}{4}}
    \end{split}
\end{equation*}

\end{multicols}
%%%%%%%%%%%%%%%%%%%%%%%%%%%%%%%%%%%%%%%%%%%%%%%%%%%%%%%%%%%%%%%%%%%%%%%%%%%%%

\begin{bottompar}
    {\footnotesize \ttfamily (CC BY-SA 4.0) 2019 Błażej Sewera \par
    src: https://github.com/jazzsewera/pofa-wzory}
\end{bottompar}

%%%%%%%%%%%%%%%%%%%%%%%%%%%%%%%%%%%%%%%%%%%%%%%%%%%%%%%%%%%%%%%%%%%%%%%%%%%%%
%%%%%%%%%%%%%%%%%%%%%%%%%%%%%%%%%%%%%%%%%%%%%%%%%%%%%%%%%%%%%%%%%%%%%%%%%%%%%
%%%%%%%%%%%%%%%%%%%%%%%%%%%%%%%%%%%%%%%%%%%%%%%%%%%%%%%%%%%%%%%%%%%%%%%%%%%%%

\newpage

\begin{multicols}{3}

\begin{equation*}
    \Gamma = \frac{Z_2 - Z_1}{Z_2 + Z_1}
\end{equation*}
\begin{equation*}
    \mathrm{WFS} = \frac{1 + |\Gamma|}{1 - |\Gamma|}
\end{equation*}

\grayrule

\begin{equation*}
    \begin{split}
        v_p &= \frac{\omega}{\beta} \\
        v_g &= \frac{\partial \omega}{\partial \beta} \\
        v_p \cdot v_g &= c^2
    \end{split}
\end{equation*}
\begin{equation*}
    \delta_p = \frac{1}{\alpha} \approx \sqrt{\frac{2}{\omega \mu \sigma}}
\end{equation*}

\grayrule

\begin{equation*}
    \begin{split}
        w &= w_E + w_H \\
        w_E &= \frac{1}{2} \left(\Vec{E}\cdot\Vec{D}\right) \\
        w_H &= \frac{1}{2} \left(\Vec{H}\cdot\Vec{B}\right) \\
    \end{split}
\end{equation*}
\begin{equation*}
    \Vec{S} = \Vec{E} \times \Vec{H}
\end{equation*}

\grayrule

\begin{equation*}
    \begin{split}
        E_{-} &= E_{ox} \Vec{i_x} \cos(\omega t - \beta z) \\
              &+ E_{oy} \Vec{i_y} \cos(\omega t - \beta z)
    \end{split}
\end{equation*}
\grayrule
\begin{equation*}
    \begin{split}
        E_{\bigcirc} &= E_{o} \Vec{i_x} \cos(\omega t - \beta z) \\
                     &+ E_{o} \Vec{i_y} \sin(\omega t - \beta z)
    \end{split}
\end{equation*}
\grayrule
\begin{equation*}
    \begin{split}
        E_{O} &= E_{ox} \Vec{i_x} \cos(\omega t - \beta z) \\
              &+ E_{oy} \Vec{i_y} \sin(\omega t - \beta z)
    \end{split}
\end{equation*}

\grayrule

\begin{equation*}
    \begin{split}
        T_E &= 1 + \Gamma \\
        T_H &= 1 - \Gamma
    \end{split}
\end{equation*}

\grayrule
\begin{equation*}
    \begin{split}
        &Z_{f2 \, tr}(z = -d) \\
        &= Z_{w2} \cdot \frac{e^{\gamma_2 d} + \Gamma_{23} e^{-\gamma_2 d}}{e^{\gamma_2 d} + \Gamma_{23} e^{-\gamma_2 d}}
    \end{split}
\end{equation*}

\begin{equation*}
    \begin{split}
        &[\sigma_2 = 0] \implies \\
        &Z_{f2 \, tr}(z = -d) \\
        &= Z_{w2} \cdot \frac{Z_{w3} + j Z_{w2} \tan(\beta_2 d)}{Z_{w2} + j Z_{w3} \tan(\beta_2 d)}
    \end{split}
\end{equation*}

\begin{equation*}
    \begin{split}
        \Gamma_{12} &= \frac{Z_{f2 \, tr} - Z_{w1}}{Z_{f2 \, tr} + Z_{w1}} \\
        \Gamma_{23} &= \frac{Z_{w3} - Z_{w2}}{Z_{w3} + Z_{w2}}
    \end{split}
\end{equation*}

\grayrule

\begin{equation*}
    \begin{split}
        d = \frac{\lambda}{4} \, &\wedge \, Z_{w2} = \sqrt{Z_{w1} \cdot Z_{w3}} \\
                                                  &\Downarrow \\
        \Gamma_{12} &= 0
    \end{split}
\end{equation*}

\end{multicols}

\begin{bottompar}
    {\footnotesize \ttfamily (CC BY-SA 4.0) 2019 Błażej Sewera \par
    src: https://github.com/jazzsewera/pofa-wzory}
\end{bottompar}

%%%%%%%%%%%%%%%%%%%%%%%%%%%%%%%%%%%%%%%%
%Do not alter anything below this line.
\end{document}
